\documentclass{article}
\usepackage{graphicx} % Required for inserting images
\usepackage{amsmath}
\usepackage{breqn}
\usepackage{xcolor}
\usepackage{tikz}
\usepackage{tcolorbox}
\usepackage{comment}
\usepackage{subcaption}
\usepackage[utf8]{inputenc}
\usepackage{float}
\usepackage{placeins}

\title{Retirer second membre à haute fréquence}
\author{Hervé SV}
\date{Juin 2023}

\begin{document}

\maketitle

On se donne un version simplifié et linéairisé de l'équation pour l'oscillateur de Duffing :

\[ \dot{z}(t) -i (\omega_0 - \omega)z + \gamma_m z = \frac{f_0}{2} (1 + e^{-i2\omega t}) \]

On la sépare en deux

\begin{dmath}
    \label{eq:1}
    \dot{z}(t) -i (\omega_0 - \omega)z + \gamma_m z = \frac{f_0}{2}e^{-i2\omega t}
\end{dmath}

\begin{dmath}
    \label{eq:2}
    \dot{z}(t) - i (\omega_0 - \omega)z + \gamma_m z = \frac{f_0}{2}
\end{dmath}

On suppose que $z_1(t)$ (solution de (\ref{eq:1})) soit de la forme :
\[ z_1(t) = Ae^{-i2wt} \]
Donc l'équation devient :
\[ (-i2\omega - i(\omega_0 - \omega) + \gamma_m)A = \frac{f_0}{2} \]
Où
\[ A = \frac{1}{-i2\omega - i(\omega_0 - \omega) + \gamma_m}\frac{f_0}{2}  \]

L'équation (\ref{eq:2}) est beaucoup plus simple, on prend $z_2(t)$ constant
\[ z_2(t) = A' \]
\[ (- i(\omega_0 - \omega) + \gamma_m)A' = \frac{f_0}{2} \]
\[ A' = \frac{1}{- i(\omega_0 - \omega) + \gamma_m}\frac{f_0}{2}  \]

Dans le cas où $\omega >> \gamma_m$ et que $\omega \approx \omega_0$, on a aussi $\omega >> (\omega_0 - \omega)$. Donc :
\[ |A| \approx \frac{1}{i2\omega}\frac{f_0}{2}  << \frac{1}{- i(\omega_0 - \omega) + \gamma_m}\frac{f_0}{2} = A' \]

\[ |\frac{A}{A'}| \approx |\frac{- i(\omega_0 - \omega) + \gamma_m}{i2\omega}|  << 1 \]

$z_1(t)$ est donc négligeable par rapport à $z_2(t)$, ce qu'on peut prendre ne compte dans l'équation d'origine en éliminant le terme $\frac{f_0}{2}e^{-i2\omega t}$










\end{document}