\documentclass{report}

\usepackage{graphicx} % Required for inserting images
\usepackage[french]{babel}
\usepackage[T1]{fontenc}
\usepackage{geometry}
\usepackage{setspace}

\usepackage[font=small,labelfont=bf]{caption}
\usepackage{subcaption}
\usepackage[font=small,labelfont=bf]{caption}
\usepackage{multicol}
\usepackage{titlesec}

\usepackage[utf8]{inputenc}
\usepackage[
            backend=biber,
            style=numeric,
            %langid=french,
            sorting=none,
            ]{biblatex}
\DeclareNameAlias{author}{last-first}

% to ensure citations follow french convention
\usepackage{csquotes}


\usepackage{breqn}
\usepackage{xcolor}
\usepackage{tikz}
\usepackage{tcolorbox}
\usepackage{comment}


\usepackage[hidelinks, colorlinks=true, allcolors=blue]{hyperref}


\usepackage{float}

\edef\restoreparindent{\parindent=\the\parindent\relax}
\usepackage{parskip}

\captionsetup{belowskip=0pt}


\usepackage{amsmath}
%\usepackage{amsfonts}
\usepackage{siunitx}
\usepackage{physics}


\geometry{a4paper, margin=1.9cm}

% removes large overhead margin from chapter header
\titlespacing{\chapter}{0pt}{-30pt}{10pt}

% removes "Chapter N" text
\titleformat{\chapter}[display]
  {\huge\bfseries}
  {}
  {0pt}
  {\thechapter.\ }


\titleformat{name=\chapter,numberless}[display]
  {\huge\bfseries}
  {}
  {0pt}
  {}






\begin{document}

Lalalala

\begin{align}
    \label{eq:O_1}
    x_0'' + x_0 &= 0  \\
    \label{eq:O_eps}
    x_1'' + x_1 &= -2\omega_1 x_0'' - (x_0^2 - 1)x_0' \\
    \label{eq:O_eps2}
    x_2'' + x_2 &= -2\omega_2 x_1'' - (2\omega_2 + \omega_1^2)x_0'' - (x_0^2 - 1)x_1' - 2x_0 x_1 x_0' - \omega_1(x_0^2 - 1)x_0'
\end{align}

%A cause de la periodicité de notre solution, on peut, sans perdre de géneralité, imposer les conditions initiales suivantes.

On peut constater que si on impose des conditions initiales arbitraires $x(0, \epsilon)=A,  \, x'(0, \epsilon)=B$, pour que \eqref{eq:x_eps} soit valide pour tout $\epsilon$ il faut obligatoirement que :
\begin{equation}
    x_0(0) = A,\, x_0'(0) = B
    \qquad
    x_k(0) = x_k'(0) = 0 \; \forall k > 0
\end{equation}

On commence donc par résoudre \eqref{eq:O_1} yeet

\end{document}